\chapter{Los números racionales y los números reales}

\section{Los números racionales}

En el conjunto $\mathbb{Z} \times \mathbb{Z}^*$ se define la relación de equivalencia $\mathcal{E}$ mediante:
\[
(a,b) \mathcal{E} (a',b') \Leftrightarrow ab'=a'b
\]

Toda clase de equivalencia es por definición un \textbf{número raciona}l y el conjunto de todas las clases de equivalencia o conjunto conciente $\mathbb{Z} \times \mathbb{Z}^*/\mathcal{E}$ es el conjunto de números racionales y se denota $\mathbb{Q}$.

Si $(a,b) \in \mathbb{Z}^* \times \mathbb{Z}^*$ y $d = mcd(|a|,|b|)$ entonces $a=da'$ y $b=db'$, siendo $mcd(|a'|,|b'|)=1$. Se denomina $(a',b')$ \textbf{representante canónico} o \textbf{fracción irreducible}. Se elige generalmente con $b \in \mathbb{N}^*$.

\subsection{Operaciones en $\mathbb{Q}$}

Sean $\alpha,\beta in \mathbb{Q}$ y sean $(a,b),(c,d) \in \mathbb{Z} \times \mathbb{Z}^*$ sus representantes. Se define:
\begin{align*}
\alpha + \beta &= [(ad+cb,bd)] = \frac{ad+cb}{bd} \\
\alpha\beta &= [(ac,bd)] = \frac{ac}{bd}
\end{align*}

$(\mathbb{Q},+,\cdot)$es un cuerpo.

\subsection{Orden en $\mathbb{Q}$}

Dados $\alpha,\beta \in \mathbb{Q}$, se define la relación:
\[
\alpha \leq \beta \Leftrightarrow \beta - \alpha \in \mathbb{Q}_+
\]

Tiene las siguientes propiedades:
\begin{itemize}
	\item Reflexiva.
	\item Antisimétrica.
	\item Transitiva.
	\item Es de orden total.
	\item Es compatible con la suma.
\end{itemize}

$(\mathbb{Q},+,\cdot,\leq)$ es un cuerpo ordenado.

\subsection{Propiedad arquimediana de $\mathbb{Q}$}

Dados $\alpha,\beta \in \mathbb{Q}$ con $\alpha > 0 $, existe un $n \in \mathbb{N}$ tal que $n\alpha > \beta$.

\subsection{Orden divisible}

El orden de $\mathbb{Q}$ es \textbf{divisible}, es decir, para todo $\alpha,\beta \in \mathbb{Q}$ tales que $alpha < \beta$, existe $\gamma \in \mathbb{Q}$ tal que $\alpha < \gamma < \beta$.

$\mathbb{Z}$ no tiene esta propiedad y  por eso se dice que el orden de $\mathbb{Z}$ es discreto.

\subsection{Los números decimales}

Un número racional forma parte del conjunto $\mathbb{D}$ de los números decimales si y sólo si el denominador de su fracción irreducible es de la forma $2^n5^p$ con $n,p \in \mathbb{N}$.

Para todo $n \in \mathbb{N}$, existe un único $c \in \mathbb{N}$ que cumple
\[
\frac{c}{10^n} \leq \frac{a}{b} < \frac{c+1}{10^n}
\]
Siendo $c$ el cociente de dividir $a10^n$ por $b$.