\chapter{Operaciones internas y estructuras algebraicas}

\section{Operaciones internas}

Sea $E$ un conjunto. Una \textbf{operación interna}, o \textbf{ley de composición interna}, en $E$ es una aplicación de $E \times E$ en $E$. Es decir, es una ley que asocia a todo par $(a,b)$ de elemento de $E$ un elemento único de $E$, que notaremos, $a \star b$.

\subsection{Propiedades}

Sea $E$ un conjunto y $\star$ una operación interna definida en $E$:

\begin{description}
\item[Asociativa]
\[
\forall a,b,c \in E; a \star (b \star c) = (a \star b) \star c
\]
\item[Conmutativa] 
\[
\forall a,b \in E; a \star b = b \star a
\]
Los elementos que cumplen esta propiedad se llaman \textbf{conmutables}.
\item[Elemento neutro] 
\[
\exists e \in E,\forall a \in E; a \star e = e \star a = a
\]
Si existe elemento neutro de $\star$ en $E$ este es único.
\item[Elemento simétrico] del elemento $a \in E$ a un elemento $a' \in E$ tal que $a \star a' = a' \star a = e$.

Sea $\star$ una operación asociativa interna en $E$ con elemento neutro $e \in E$. Si $a \in E$ tiene elemento simétrico, este es único.
\end{description}

\section{Grupos}

Sean $G$ un conjunto y $\star$ una \textbf{operación interna} en $G$. Se dice que el par $(G,\star)$ tiene estructura de grupo, o que $(G,\star)$ es un \textbf{grupo}, si se satisfacen las siguientes propiedades:
\begin{enumerate}
\item $\star$ es \textbf{asociativa}.
\item Existe \textbf{elemento neutro} de $\star$ en $G$.
\item Para todo elemento $a \in G$, existe en $G$ el \textbf{elemento simétrico} de $a$ respecto de $\star$.
\end{enumerate}

Si ademas la operación $\star$ es conmutativa se dice que el grupo es \textbf{conmutativo} o \textbf{abeliano}.

\subsection{Propiedades de un grupo}

Sea $(G,\star)$ un grupo. Se tiene:
\begin{enumerate}
\item $\forall a,b \in G, a \star b = a \star c \Rightarrow b = c$ (Propiedad cancelativa).
\item Para todo $a,b \in G$, existe un único $x \in G$, tal que $a \star x = b$.
\item Si $a^{-1}$ y $b^{-1}$ son los simétricos de $a$ y $b$ respectivamente, entonces ${(a \star b)}^{-1} = b^{-1} \star a^{-1}$.
\end{enumerate}

\subsection{Subgrupos}

Dados $(G,\star)$ y $H \subset G$, se dice que $H$ es un subgrupo de $G$ si $(H,\star)$ tiene a su vez estructura de grupo. En particular, el subconjunto $\{e\}$ y el propio $G$ son subgrupos de $G$.

Sean un grupo $(G,\star)$ y $H$ un subconjunto no vacío de $G$, $(H,\star)$ es un subgrupo de $G$ si y sólo si $\forall a,b \in H, a \star b^{-1} \in H$.

\subsection{Congruencia modulo}

Sea $(G,\star)$ un \textbf{grupo abeliano} y sea $(H,\star)$ un subgrupo. La relación $\mathcal{R}_H$ en $G$ definida:
\[
\forall a,b \in G, a \mathcal{R}_H b \Leftrightarrow a \star b^{-1} \in H
\]
es una relación de equivalencia, que se denomina \textbf{congruencia modulo H}.

Las clases de equivalencia generadas son equipotentes en $G$. Y si $G$ es finito $card([a])|card(G)$.

\section{Anillos}

Sea $A$ un conjunto y sean $+$ y $\cdot$ dos operaciones internas definidas en $A$. Diremos que $(A,+,\cdot)$ es un \textbf{anillo} si se satisface:
\begin{enumerate}
\item $(A,+)$ es un grupo conmutativo.
\item La operación $\cdot$ es asociativa.
\item La operación $\cdot$ es distributiva respecto de la operación $+$, esto es,
\begin{align*}
a(b+c)=ab+ac \\ (b+c)a=ba+ca
\end{align*}
\end{enumerate}

Si además la operación $\cdot$ es conmutativa, se dice que $(A,+,\cdot)$ es una \textbf{anillo conmutativo}.

Si $(A,+,\cdot)$  tiene elemento neutro para $\cdot$, siendo este distinto del elemento neutro de $\star$, se dice que $(A,+,\cdot)$  es una \textbf{anillo unitario}.

\subsection{Propiedades de un anillo}

Sea $(A,+,\cdot)$ un anillo. Se tiene:
\begin{enumerate}
\item $\forall a \in A, A \cdot 0 = 0 \cdot A = 0$, se dice que $0$ es absorbente para el producto.
\item $\forall a,b \in A, (-a)b=a(-b)=-(ab)$ y $(-a)(-b)=ab$.
\item Si además el $(A,+,\cdot)$ es una anillo conmutativo se satisfacen las siguientes igualdades:
\begin{align*}
{(a+b)}^2&=a^2+b^2+2ab \\
(a+b)(a-b) &= a^2-b^2 \\
{(a+b)}^n&=\sum_{i=0}^n \begin{pmatrix}
n \\ i
\end{pmatrix} a^{n-i}b^i
\end{align*}
\end{enumerate}

\subsection{Divisores de cero}

En un anillo $(A,+,\cdot)$ se dice que el elemento $a \in A, a \neq 0$, es un divisor de cero si existe $b \in A, b \neq 0$, tal que $ab=0$. Un anillo que no tenga divisores de cero se llama \textbf{anillo integro}.

\subsection{Subanillos. Ideales}

Sea $(A,+,\cdot)$ un anillo y $H$ un subconjunto no vació de $G$. Se dice que $H$ es un \textbf{subanillo} de A si $(H,+,\cdot)$ es a su vez una anillo. Los subconjuntos $\{1\}$ y el propio $A$ son subanillos de $A$.

Sean $(A,+,\cdot)$ un anillo y $H$ un subconjunto no vacío de $A$. $H$ es un \textbf{subanillo} de $A$ si y sólo si $\forall a,b \in H$ se cumple:
\begin{enumerate}
\item $a-b \in H$.
\item $ab \in H$.
\end{enumerate}

Sean $(A,+,\cdot)$ un \textbf{anillo conmutativo} e $I$ un subconjunto no vacío de $A$. Se dice que $I$ es un \textbf{ideal} de $A$ si cumple:
\begin{enumerate}
\item $a-b \in I, \forall a,b \in I$.
\item $ac \in I, \forall a \in I, \forall c \in A$.
\end{enumerate}

Si $(A,+,\cdot)$ un \textbf{anillo conmutativo} y $a \in A$ es un elemento fijo, el conjunto
\[
aA = (a) = \{ak | k \in A\}
\]
es un ideal de $A$ que se denomina \textbf{ideal principal} generado por $a$.

\section{Cuerpos}

Sea $\mathbb{K}$ un conjunto y seán $+$ y $\cdot$ dos operaciones internas definidas en $\mathbb{K}$. $(\mathbb{K},+,\cdot)$ es un \textbf{cuerpo} si se satisfacen las siguientes propiedades:
\begin{enumerate}
  \item Las operaciones $+$ y $\cdot$ son asociativas en $\mathbb{K}$.
  \item Las operaciones $+$ y $\cdot$ son conmutativas en $\mathbb{K}$.
  \item La operación $\cdot$ es distributiva respecto la operación $+$ en $\mathbb{K}$.
  \item Existen dos elementos distintos en $\mathbb{K}$ que se designan por $0,1$ que son elementos neutros de la suma y del producto respectivamente.
  \item Existencia de opuesto: para todo $a$ en $\mathbb{K}$ existe el simétrico de $a$ respecto de la suma que se designa por $-a$.
  \item Existencia de inverso: para todo elemento $a \neq 0$ de $\mathbb{K}$ existe el simétrico de $a$ para el producto que se designa por $a^{-1}$.
\end{enumerate}
O lo que es lo mismo:
\begin{enumerate}
  \item $(\mathbb{K},+)$ es un grupo conmutativo.
  \item $(\mathbb{K},\cdot)$ es un grupo conmutativo.
  \item La operación $\cdot$ es distributiva respecto de la operación $+$ en $\mathbb{K}$.
\end{enumerate}

Un cuerpo no tiene divisores de cero, es decir, es un anillo integro.

\subsection{Subcuerpos}

Sea $(\mathbb{K},+,\cdot)$ un cuerpo y sea $H$ un subconjunto no vacío de $\mathbb{K}$ donde consideramos las restricciones de las operaciones en $\mathbb{K}$. Se dice que $H$ es un \textbf{subcuerpo} de $\mathbb{K}$ si $(H,+,\cdot)$ es su vez un cuerpo.

Sea $(\mathbb{K},+,\cdot)$ un cuerpo y sea $H$ un subconjunto con al menos dos elementos distintos. $H$ es un subcuerpo de $\mathbb{K}$ si y sólo si se cumple:
\begin{enumerate}
  \item $a-b \in H$ para todo $a,b \in H$.
  \item $ab^{-1} \in H$ para todo $a,b \in H^*=H\backslash \{0\}$.
\end{enumerate}

\section{Orden y operaciones}

\subsection{Grupo ordenado}

Sea una una relación de orden $\preceq$ definida sobre un grupo conmutativo $(G,+)$, se dice que $(G,+,\preceq)$ es un \textbf{grupo ordenado} si la relación de orden es compatible con la suma, es decir:
\[
\forall a,b,c \in G, a \preceq b \Rightarrow a+c \preceq b+c
\]

En un grupo ordenado $(G,+,\preceq)$ se satisfacen las siguientes propiedades:
\begin{enumerate}
  \item $a \preceq b \Leftrightarrow b+(-a) \in G_+$.
  \item $a \preceq b \wedge a' \preceq b' \Rightarrow a+a' \preceq b+b'$.
  \item $a \preceq b \Rightarrow -b \preceq -a$.
\end{enumerate}

\subsection{Anillo ordenado}

Sea una una relación de orden $\preceq$ definida sobre un grupo conmutativo $(A,+,\cdot)$, se dice que $(A,+,\cdot,\preceq)$ es un \textbf{anillo ordenado} si se cumple:
\begin{enumerate}
  \item $\forall a,b,c \in A, a \preceq b \Rightarrow a+c \preceq b+c$.
  \item $\forall a,b \in A, 0 \preceq a \wedge 0 \preceq b \Rightarrow 0 \preceq ab$.
\end{enumerate}

Si la relación de orden es total, se dice que el anillo es una \textbf{anillo totalmente ordenado}. Si además, es un cuerpo hablaremos de \textbf{cuerpo ordenado}.

En un anillo totalmente ordenado se define el \textbf{valor absoluto} de $a \in A$ mediante:
\[
|a|=\begin{cases}
a &si \quad 0 \preceq a \\
-a & si \quad a \prec a
\end{cases}
\]

En un anillo totalmente ordenado $(A,+,\cdot,\preceq)$ se satisfacen las siguientes propiedades:
\begin{enumerate}
	\item $a \preceq b \Leftrightarrow b-a \in A$.
	\item $a \preceq b \wedge a' \preceq b' \Rightarrow a+a' \preceq b+b'$.
	\item $a \preceq b \Rightarrow -b \preceq -a$.
	\item $a \preceq b \wedge 0 \preceq c \Rightarrow ac \preceq bc$.
	\item $a \preceq b \wedge c \preceq 0 \Rightarrow bc \preceq ac$.
	\item $\forall a \in A, 0 \preceq a^2$.
	\item Si $A$ es un anillo unitario entonces $0 \prec 1$.
	\item $|a| \succeq 0, \forall a \in A$ y $|a| = 0 \Leftrightarrow a=0$.
	\item $|ab|=|a||b|, \forall a,b \in A$.
	\item $|a+b| \preceq |a|+|b|, \forall a,b \in A$.
\end{enumerate}

Si además $(A,+,\cdot)$ es un cuerpo también se cumple:
\begin{enumerate}
	\item $a \succ 0 \Rightarrow a^{-1} \succ 0$.
	\item $0 \prec a \preceq b \Rightarrow b^{-1} \preceq a^{-1}$.
	\item $a \preceq b \prec 0 \Rightarrow b^{-1} \preceq a^{-1}$.
\end{enumerate}

\section{Homomorfismos}

Sean $G$ y $G'$ dos conjuntos donde se tiene respectivamente definida una operación interna $+$. Sea $f: G \longrightarrow G'$ una aplicación. Se dice que $f$ es un \textbf{homomorfismo} si se cumple que:
\[
\forall a,b \in G, f(a+b) = f(a) + f(b)
\]

Si $G=G'$ se denomina \textbf{endomorfismo}. Si el homomorfismo es biyectivo hablaremos de \textbf{isomorfismo} y todo endomorfismo biyectivo se denomina \textbf{automorfismo}.

\subsection{Propiedades de un homomorfismo}

\begin{enumerate}
	\item Si $f: G \longrightarrow G'$ es un homomorfismo entonces la operación de $G'$ es una operación interna cuando se restringe al conjunto imagen $f(G)$.
	\item Si $f: G \longrightarrow G'$ y $g: G' \longrightarrow G''$ son homomorfismo entonces la composición $g \circ f: G \longrightarrow G'$ es un homomorfismo.
	\item Si $f: G \longrightarrow G'$ es un isomorfismo entonces la aplicación inversa $f^{-1}: G' \longrightarrow G$ es un isomorfismo.
\end{enumerate}

La existencia de un isomorfismo entre dos conjuntos define una relación de equivalencia, ya que es reflexiva, simétrica y transitiva.

\subsection{Homomorfismo de grupo}

Sean $(G,+)$ y $(G',+)$ dos grupos tales que sus elementos neutros son respectivamente $0_G$ y $0_{G'}$, y $-a$ y $-a'$ los elementos simétricos de $a \in G$ y $a' \in G'$. Sea $f: G \longrightarrow G'$. Se tiene:
\begin{enumerate}
	\item $f(0_G)=0_{G'}$.
	\item $f(-a)=-f(a), \forall a \in G$.
	\item Si $H$ es un subgrupo de $G$ entonces,
	\[
	f(H)=\{a' \in G'| \exists a \in H, f(a)=a'\}
	\]
	es un subgrupo de $G'$.
	\item Si $H'$ es un subgrupo de $G'$ entonces,
	\[
	f^{-1}(H')=\{a \in G| a \in G, f(a) \in H'\}
	\]
	es un subgrupo de $G$.
\end{enumerate}

Sean $(G,+)$ y $(G',+)$ dos grupos y $f: G \longrightarrow G'$ es un homomorfismo. Se tiene:
\begin{enumerate}
	\item $Im f$ es un subgrupo de $G'$.
	\item $Ker f$ es un subgrupo de $G$.
	\item $f$ es inyectivo si y sólo si $Ker f = \{0_G\}$.
	\item $f$ es sobreyectivo si y sólo si $Im f = G'$
\end{enumerate}

\subsection{Homomorfismo de anillos y cuerpos}

Si $(A,+,\cdot)$ y $(A',+,\cdot)$ son dos anillos, un \textbf{homomorfismo de anillos} es una aplicación $f: A \longrightarrow A'$ tal que para todo $a,b \in A$ se cumple:
\begin{enumerate}
	\item $f(a+b)=f(a)+f(b)$.
	\item $f(ab)=f(a)f(b)$.
\end{enumerate}

Para la $+$ se cumplen las todas propiedades del homomorfismo de grupo, y para $\cdot$ las propiedades de homomorfismo.

Si Si $(A,+,\cdot)$ y $(A',+,\cdot)$ son cuerpos estaríamos ante un \textbf{homomorfismo de cuerpos}.

\subsection{Homomorfismo de conjuntos ordenados}

Si tenemos dos conjuntos ordenados $(U,\preceq)$ y $(V,\preccurlyeq)$ una aplicación $f: U \longrightarrow V$ se denomina \textbf{homomorfismo de estructuras ordenadas} si es creciente, es decir:
\[
\forall a,b \in U, a \preceq b \Rightarrow f(a) \preceq f(b)
\]

Si $f$ es biyectiva estaremos ante un \textbf{isomorfirmos de estructuras ordenadas}.