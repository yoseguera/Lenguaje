\chapter{Conjuntos}

\section{Algunas ideas sobre conjuntos. Predicados}

Un conjunto $C$ está bien definido cuando se tiene un criterio que permite decidir si un determinado elemento $b$ pertenece al conjunto $C$ o no pertenece al conjunto $C$.

Un objeto no puede ser a la vez un conjunto y un elemento de ese conjunto. Este decir, la proposición $b \in b$ es falsa.

La colección de todos los conjuntos posibles no forman un conjunto.

\subsection{Igualdad de conjuntos}

Se dice que dos conjuntos $A$ y $B$ son iguales, y se escribe $A=B$, si y sólo si tiene los mismo elementos.

Cuando un conjunto se determina mediante una lista de todos sus elementos se dice que está \textbf{definido por extensión}.

\subsection{Inclusión de conjuntos}

\[
A \subset B \Leftrightarrow x \in A \wedge x \in B
\]

\[
A = B \Leftrightarrow A \subset B \wedge B \subset A
\]

$A$ y $\emptyset$ son \textbf{subconjuntos impropios} de $A$.

\subsection{Predicados}

\[
C_P = \{ x \in C | P_x \}
\]

$P_x$ es un predicado que indica si el elemento x forma parte del conjunto o no. Los conjuntos así definidos se dice que está \textbf{definido por inclusión}.

Al conjunto $C$ se le conoce como \textbf{universo de predicado}.

Dos predicados son \textbf{equivalentes} sobre un universo $C$ si definen el mismo subconjunto de $C$.

\subsection{Conjunto vacío}

\[
\emptyset = \{ x \in C | x \notin C \}
\]

No tiene ningún elemento y es subconjunto de cualquier conjunto.

\subsection{Principio de inducción}

Si un predicado $P$ se define sobre $\mathbb{N}$ tal que:
\begin{enumerate}
	\item $0$ satisface la propiedad $P$. Es decir, $P_x$ es verdadero.
	\item Si $n$ satisface la propiedad $P$ entonces el sucesor de $n$ satisface también la propiedad $P$.
\end{enumerate}

\subsection{Cuantificadores}

\begin{align*}
\neg (\forall x P_x) \Leftrightarrow \exists x \neg P_x \\
\neg (\exists x P_x) \Leftrightarrow \forall x \neg P_x
\end{align*}

\subsection{Complementario y partes de un conjunto}

\subsection{Complementario}

\[
\overline{A} = \{x \in U | x \notin A\} = \{x \in U | \neg P_x\}
\]
\[
A \subset B \Rightarrow \overline{B} \subset \overline{A}
\]

\subsection{Partes de un conjunto}

\[
\mathcal{P}(A) = \{ B | B \subset A \}
\]
\[
card(\mathcal{P}(A))=2^{card(A)}
\]

\section{Operaciones con conjuntos}

\subsection{Unión}

\[
A \cup B = \{x | x \in A \vee x \in B\}
\]

\subsection{Intersección}

\[
A \cap B = \{x | x \in A \wedge x \in B\}
\]

\subsection{Familia de conjuntos}

Si tenemos en cuenta un \textbf{conjunto de indices} no vacío $I$. A cada $i \in I$ le asociamos un conjunto $F_i$. La colección de todos esos conjuntos se denomina \textbf{familia de conjuntos} y se denota:
\[
F = \{F_i | i \in I\}
\]

Cuando todos los $F_i$ son subconjuntos de un mismo conjunto $U$ entonces $F$ es un subconjunto de $P(U)$. Cualquier subconjunto $G$ no vacío de $P(U)$ es una familia de conjuntos.

La unión e intersección se generaliza a familias arbitrarias:
\begin{align*}
\bigcup_{i \in I} = \{x | \exists i \in I, x \exists F_i \} \\
\bigcap_{i \in I} = \{x | \forall i \in I, x \exists F_i \} \\
\end{align*}

Si la familia viene dada por $\emptyset  \neq G \subset P(U)$:
\begin{align*}
\bigcup_{F \in G} = \{x | \exists F \in G, x \exists F_i \} \\
\bigcap_{F \in G} = \{x | \forall F \in G, x \exists F_i \} \\
\end{align*}

\subsection{Diferencia de conjuntos}

\begin{align*}
A \setminus   B = \{x | x \in A \wedge x \notin B \} = A \cap \overline{B} 
\end{align*}


La diferencia de conjuntos es distributiva:
\begin{align*}
A \setminus  (B \cap C) &= (A \setminus  B) \cap (A \setminus C) \\
A \setminus  (B \cup C) &= (A \setminus B) \cup (A \setminus C) 
\end{align*}

\subsection{Diferencia simétrica}

\begin{align*}
A \triangle B = & (A \cup B) \setminus{} (A \cap B) = \\ 
& (A \setminus  \overline{B}) \cup (\overline{A} \setminus  B) = \\ 
& \{ x | x \in A \wedge x \notin B\} \cup \{ x | x \notin A \wedge x \in B\}
\end{align*}



\section{Producto de conjuntos}

Dado $n$ conjuntos $A_1,A_2,\ldots,A_n$ se denomina producto de $A_1$, por $A_2$, por \ldots, por $A_n$ al conjunto:

\[
A_1 \times A_2 \times \cdots \times A_n = \{(x_1,x_2,\ldots,x_n)|x_1 \in A_1, x_2 \in A_2,\ldots,x_n \in A_n\}
\]

Es distributivo:
\begin{align*}
A \times (B \cap C) &= (A \times B) \cap (A \times C) \\
(B \cap C) \times A &= (B \times A) \cap (C \times A) \\
A \times (B \cup C) &= (A \times B) \cup (A \times C) \\
(B \cup C) \times A &= (B \times A) \cup (C \times A) \\
\end{align*}

\section{Relaciones entre conjuntos}

Dados los conjuntos $A$ y $B$, todo subconjunto $R \subset A \times B$ es una relación del conjunto $A$ al conjunto $B$, relación entre $A$ y $B$ o correspondencia entre $A$ y $B$.

\[
R: A \longrightarrow B \\
\]

Otra notación posible es:
\[
xRy \Leftrightarrow (x,y) \in R \subset A \times B
\]

$A$ es el conjunto inicial, y $B$ el conjunto final.

\subsection{Conjunto original de la relación $R$}

\[
R^{-1}(B)=\{x \in A | \exists y \in B, xRy\}
\]

\subsection{Conjunto final de la relación $R$}

\[
R(A)=\{y \in B | \exists x \in A, xRy\}
\]

\subsection{Conjunto imagen de $x \in A$}

\[
R(x)=\{y \in B | (x,y) \in R\} = \{y \in B | xRy\}
\]

\subsection{Conjunto original de $y \in B$}

\[
R^{-1}(y)=\{x \in A | (x,y) \in R\} = \{x \in A | xRy\}
\]

\subsection{Composición de relaciones}

Dadas las relaciones $R$ entre los conjuntos $A$ y $B$, y la relación $S$ entre los conjuntos $B$ y $C$, se define una relación entre los conjuntos $A$ y $C$ conocida como \textbf{composición de las relaciones $R$ y $S$} o relación composición:
\[
S \circ R = \{(x,z) \in A \times C | \exists y \in B, xRy \wedge ySz\}
\]

