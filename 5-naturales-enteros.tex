\chapter{Los números naturales y los números enteros}

\section{Los números naturales}

Los números naturales se definen mediante los axiomas de Peano:
\begin{enumerate}
 \item El elemento 0 es un número natural.
 \item Todo elemento natural $n$ tiene un único elemento sucesor que tambien es
un número natural.
 \item 0 no es el sucesor de ningún número natural.
 \item Dos números naturales cuyos sucesores sean iguales, son iguales.
 \item Si un conjunto de número naturales incluye al 0 y a todos los sucesores
de cada uno de los elementos, incluye a todos los números naturales.
\end{enumerate}

Se denotan de la siguiente manera, teniendo en cuenta si incluyen al 0 o no:
\begin{align*}
 \mathbb{N}=\{0,1,2,3,\ldots\} \\
 \mathbb{N}*=\{1,2,3,\ldots\}
\end{align*}

En resumen:
\begin{itemize}
  \item Todo número natural $n\neq0$ es sucesor de algún número natural.
  \item Para todo $n \in \mathbb{N}, n \neq s(n)$.
\end{itemize}

\subsection{Suma}

Se define mediante recursión sobre $n$ la suma $m+n$ mediante:
\begin{enumerate}
  \item $m+0=n$ para todo $m \in \mathbb{N}$.
  \item $m+s(n)=s(m+n)$ para todo $m,n \in \mathbb{N}$.
\end{enumerate}

La suma de números naturales es una operación interna en $\mathbb{N}$ que
satisface, cualesquiera que sean $m,n,p \in \mathbb{N}$, las siguientes
propiedades:

\begin{enumerate}
  \item Existencia de elemento neutro: $m+0=0+m=m$.
  \item Asociativa: $(m+n)+p=m+(n+p)$.
  \item Conmutativa: $m+n=n+m$.
  \item Cancelativa: $m+p=n+p \Rightarrow m=n$.
\end{enumerate}

\subsection{Producto}

Se define por recurrencia sobre $n$ el producto, que designamos por $m \cdot n$ o $mn$, de los numeros naturales $m$ y $n$ mediante:
\begin{enumerate}
	\item $m \cdot 0 = 0$ para todo $n \in \mathbb{N}$.
	\item $m \cdot s(n) = m \cdot n + m$ para todo $m, n \in \mathbb{N}$. 
\end{enumerate}

El producto de números naturales es una operación interna en $\mathbb{N}$ que satisface, cualesquiera que sean $m,n,p \in \mathbb{N}$, las siguientes propiedades:
\begin{enumerate}
	\item Existencia de elemento neutro: $m \cdot 1 = 1 \cdot m = m$.
	\item Distributiva: $m(n + p) = mn + mp$ y $(n+p)m = nm + pm$.
	\item Asociativa: $(m \cdot n) \cdot p = m \cdot (n \cdot p)$.
	\item Conmutativa: $m \cdot n = n \cdot m$.
	\item Cancelativa: $mp=np \wedge p \neq 0  \Rightarrow m=n$.
\end{enumerate}

Se define la potencia n-ésima de $a$, $a^n$, mediante:
\begin{enumerate}
	\item $0^n = 0$ para todo $n \in \mathbb{N}^*$.
	\item $a^0 = 1$ para todo $a \in \mathbb{N}^*$.
	\item $a^{n+1}=aa^n$ para todo $a \in \mathbb{N}^*$ y $n \in \mathbb{N}$.
\end{enumerate}

\subsection{Ordenación de números naturales}

Dados $m,n \in \mathbb{N}$ se define la relación \emph{menor o igual}, $\leq$, mediante:
\[
\exists p \in \mathbb{N}, m + p = n \Rightarrow m \leq n
\]

Se define \emph{estrictamente menor},$<$ como:
\[
m \leq n \wedge m \neq n \Rightarrow m < n
\]

Además, se obtiene la siguiente relación:
\[
m < n \Leftrightarrow m + 1 \leq n
\]

Las relaciones \emph{mayor o igual}, $\geq$, y \emph{estrictamente mayor}, $<$, se definen como:
\begin{align*}
m \geq n \Leftrightarrow n < m \\
m < n \Leftrightarrow n \leq m
\end{align*}

La relación $\leq$ es una \emph{relación de orden} total en $\mathbb{N}$, compatible con la suma y producto de número naturales, es decir para todo $m,n,p \in \mathbb{N}$ se tiene:
\[
m \leq n \Rightarrow \begin{cases}
m+p \leq n+p \\
mp \leq np
\end{cases}
\]

El intervalo abierto $(n, n+1)_{\mathbb{N}}$ es vacío, para todo $n \in \mathbb{N}$.

El conjunto $\mathbb{N}$ con la relación $\leq$ es un conjunto bien ordenado.

En $\mathbb{N}$, todo subconjunto no vacío y acotado superiormente, tiene máximo.

\section{Conjuntos finitos}

Un conjunto $A$ es \textbf{finito} si es vacío o si existe una biyección de $A$ sobre un listado cerrado $[1,n]_{\mathbb{N}}$, con $n \neq 0$. En caso contrario, se dice que el conjunto es \textbf{infinito}.

Los intervalos cerrados tienen las siguientes propiedades:
\begin{itemize}
	\item Si $n,m \in \mathbb{N}$ y existe una aplicación inyectiva $f: [1,n]_{\mathbb{(N)}} \longrightarrow [1,m]_{\mathbb{(N)}}$ entonces $n \leq m$.
	\item Si $n,m \in \mathbb{N}$ y existe un biyección $f: [1,n]_{\mathbb{(N)}} \longrightarrow [1,m]_{\mathbb{(N)}}$ entonces $n=m$.
\end{itemize}

Sea $A$ un conjunto finito no vacío. Existe un único número natural $n$, no nulo, tal que $A$ y $[1,n]_{\mathbb{N}}$ son equipotentes. Se dice entonces que $card(A)=n$.

Si $n \in \mathbb{N}$ entonces toda aplicación inyectiva $f: [1,n]_{\mathbb{(N)}} \longrightarrow [1,n]_{\mathbb{(N)}}$ es biyectiva.

Sea $A$ un subconjunto no vacío de $\mathbb{N}$. $A$ es un conjunto finito si y sólo si $A$ es un conjunto acotado superiormente.

Esta propiedad presenta los siguientes corolarios:
\begin{itemize}
	\item $\mathbb{N}$ es un conjunto finito.
	\item Todo subconjunto de un subconjunto finito de $\mathbb{N}$ es finito.
	\item La unión de dos subconjuntos finitos de $\mathbb{N}$ es un subconjunto finito.
	\item El complementario de un subconjunto finito de $\mathbb{N}$ es un conjunto infinito.
\end{itemize}

Sea $A$ un subconjunto de un conjunto finito de $B$ y $A \neq B$. Entonces $A$ es un conjunto finito y $card(A) < card(B)$.

Sea $A$ un conjunto finito y $f$ una aplicación de $A$ en cualquier conjunto $B$. Entonces $f(A)$ es un conjunto finito y
\[
card(f(A)) \leq card(A)
\]
Además, $card(f(A)) = card(A)$ si y sólo si $f$ es inyectiva.

Si $f$ es una aplicación sobreyectiva de un conjunto finito $A$ en un conjunto $B$, entonces:
\[
card(B) \leq card(A)
\]
Además, se tiene la igualdad $card(A)=card(B)$ si y sólo si $f$ es una aplicación biyectiva.

Sean $A$ y $B$ dos conjuntos finitos de igual cardinal y sea una aplicación $f: A \longrightarrow B$. Son equivalentes:
\begin{enumerate}
	\item $f$ es inyectiva.
	\item $f$ es biyectiva.
	\item $f$ es sobreyectiva.
\end{enumerate}

Sean $A$ y $B$ dos conjuntos finitos disjuntos. Entonces $A \cup B$ es un conjunto finito y:
\[
card(A \cup B) = card(A) + card(B)
\]

Sean $A$ y $B$ dos conjuntos finitos. Entonces $A \cup B$ y $A \cap B$ son conjuntos finitos y
\[
card(A \cup B) + card(A \cap B) = card(A) + card(B)
\]

Sean $A$ y $B$ dos conjuntos finitos. Entonces, $A \times B$ es un conjunto finito y
\[
card(A \times B) = card(A) \cdot card(B)
\]

Sean $A$ y $B$ dos conjuntos finitos. Supongamos que $a = card(A) \neq 0$ y $b = card(B) \neq 0$. Entonces
\[
card(\mathcal{F}(A,B))=card(B^A)=b^a
\]

Sea $A$ un conjunto finito. Entonces
\[
card(\mathcal{P}(A))=2^{card(A)}
\]

Sea $A$ y $B$ dos conjuntos finitos cuyos cardinales son $card(A) = a \neq 0$ y $card(B) = b \neq 0$. Entonces el número de aplicación inyectivas de $A$ en $B$ es
\[
card(\mathcal{I}(A,B)))=b(b-1)\cdots(b-a+1)
\]

El número anterior se conoce como \textbf{variaciones de $m$ sobre $n$} $V_{m,n}$:
\[
V_{m,n} = m(m-1)\cdots(m-n+1) = \frac{m!}{(m-n)!}
\]

Sea $A$ y $B$ dos conjuntos finitos cuyos cardinales son $card(A) = card(B) = n$. Entonces el número de aplicaciones biyectivas de $A$ en $B$ es:
\[
card(\mathcal{B}(A,B))= n!
\]

Sea $A$ un conjunto finito tal que $card(A)=m$. Sea $0 \leq n \leq m$. El número de subconjuntos de $A$ que poseen exactamente $n$ elementos es
\[
\begin{pmatrix}
m \\ n
\end{pmatrix}=
\frac{m!}{n!(m-n)!}
\]
Es número se lee $m sobre n$ y se denomina \textbf{coeficiente binomial}, \textbf{número combinatorio} o \textbf{combinaciones de m sobre en} $C_{m,n}$.

\section{Conjuntos infinitos}

Sea $A$ un conjunto cualquiera. Entonces el conjunto $\mathcal{P}(A)$ y el conjunto $A$ no son equipotentes.

Un conjunto $A$ se considera \textbf{numerable} si es equipotente con $\mathbb{N}$. El cardinal de los conjuntos numerables se denota como \emph{aleph sub zero} $\aleph_0$.

Sea $A$ un subconjunto de $\mathbb{N}$. Entonces $A$ es un conjunto finito o $A$ es un conjunto numerable. 

El conjunto $\mathbb{N}^2=\mathbb{N} \times \mathbb{N}$ es numerable.

Se satisfacen las siguientes propiedades:
\begin{enumerate}
	\item Todo subconjunto de un conjunto numerable es finito o numerable.
	\item El producto de conjuntos numerables es numerable.
	\item La unión de dos conjuntos numerables es numerables.
	\item La unión numerable de conjuntos numerables es numerable.
\end{enumerate}

\section{Los números enteros}

En $\mathbb{N} \times \mathbb{N}$ se define la relación de equivalencia $\mathcal{E}$:
\[
(a,b) \mathcal{E} (a',b') \Leftrightarrow a+b'=a'+b
\]

El \textbf{representante canónico del número entero $\alpha$} es el par $(m,n)$ donde al menos una de sus componentes es nula.

\subsection{Operaciones en $\mathbb{Z}$}

Sean $\alpha,\beta \in \mathbb{Z}$ y sean $(a,b),(c,d) \in \mathbb{N} \times \mathbb{N}$ sendos representantes. Se define:
\begin{align*}
\alpha + \beta &= (a+c,b+d) \\
\alpha\beta &= (ac+bd,bc+ad)
\end{align*}

$\mathbb{Z}$ es una anillo conmutativo unitario.
\begin{align*}
&\alpha \cdot 0 = 0 \cdot \alpha = 0 \\
&\alpha, \beta \in \mathbb{Z}, \alpha\beta=0 \Rightarrow \alpha = 0 \vee \beta = 0
\end{align*}

\subsection{Orden en $\mathbb{Z}$}

Dados $\alpha,\beta \in \mathbb{Z}$:
\[
\alpha \leq \beta \Leftrightarrow \beta - \alpha \in \mathbb{Z}_+
\]

$(\mathbb{Z},+,\cdot,\leqslant)$ es un anillo totalmente ordenado.

\subsection{Identificación de $\mathbb{N}$ con $\mathbb{Z}_+$}

Todo subconjunto de $\mathbb{Z}$ no vacío y acotado
\begin{enumerate}
	\item superiormente tiene máximo.
	\item inferiormente tiene mínimo.
\end{enumerate}

\subsection{Propiedad arquimediana de $\mathbb{Z}$}

Dados $\alpha,\beta \in \mathbb{Z}$ si $\alpha > 0$ entonces existe $n \in \mathbb{N}$ tal que $n\alpha > \beta$.

\section{Máximo común divisor y mínimo común múltiplo}

\subsection{División entera}

Sean $a,b \in \mathbb{Z}$ tales que $b > 0$. Existen $q,r \in \mathbb{Z}$ únicos tales que:
\[
a = qb + r \qquad y \qquad  0 \leq q < b
\]

Los números $q$ y $r$ se denominan respectivamente \textbf{cociente} y  \textbf{resto} de la división entera de $a$ entre $b$.

Se considera la relación \emph{divide} en $\mathbb{Z}$, $b$ divide a $a$, definida por:
\[
b|a \Leftrightarrow \exists q \in \mathbb{Z}, a=qb
\]

No es una relación de orden ya que no es antisimétrica. Y tiene las siguientes propiedades:
\begin{enumerate}
	\item  0 es divisible por cualquier entero.
	\item 1 y -1 son divisores de cualquier entero.
	\item $b|a \Leftrightarrow a \in b\mathbb{Z}$
	\item $b|a \Leftrightarrow a\mathbb{Z} \subset b\mathbb{Z}$
\end{enumerate}

\subsection{Mínimo común múltiplo}

Sean $a,b \in \mathbb{N}^*$. Se tiene:
\begin{enumerate}
	\item Existe un único $m \in \mathbb{N}^*$ tal que $a\mathbb{Z} \cap b\mathbb{Z} = m\mathbb{Z}$.
	\item Además, como el $m$ anterior es un múltiplo común de $a$ y $b$ y si $n \in \mathbb{Z}$ es un múltiplo común de $a$ y $b$, entonces $n$ es múltiplo de $m$.
\end{enumerate}

\subsection{Máximo común divisor}

Sean $a,b \in \mathbb{N}^*$. Se tiene:
\begin{enumerate}
	\item Existe un único $d \in \mathbb{N}^*$ tal que $a\mathbb{Z}+b\mathbb{Z}=d\mathbb{Z}$.
	\item Además, $d$ es un divisor común de $a$ y $b$ y si $n \in \mathbb{Z}$ es un divisor común de $a$ y $b$, entonces $n$ es divisor de $d$.
\end{enumerate}

\subsection{Identidad de Bézout}

Sean $a,b \in \mathbb{N}^*$ y $d = mcd(a,b)$, entonces existen $u, v \in \mathbb{Z}$ tales que:
\[
d = au+bv
\]
Además, $d$ es el mínimo número de $\mathbb{N}^*$ que se puede expresar en la forma $am+bn$ siendo $m,n \in \mathbb{Z}$.

Sean $a,b \in \mathbb{Z}^*$, se dice que $a$ y $b$ son primos entre sí si y sólo si $mcd(|a|,|b|)=1$.

\subsection{Teorema de Bézout}

Sean $a,b \in \mathbb{N}^*$. Los números $a$ y $b$ son primos entre sí y sólo sí existen $u,v \in \mathbb{Z}$ tales que $au+bv=1$.

\subsection{Teorema de Gauss}

Si $a$ y $b$ son primos entre sí y $a$ divide a $bc$ entonces $a$ divide a $c$.

Si $a,b$ y $a,c$ son primos entre sí entonces $a$ y $bc$ son primos entre sí.