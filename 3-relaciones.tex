\chapter{Relaciones y aplicaciones entre conjuntos}

\section{Propiedades básicas}

\subsection{Reflexiva}

$R$ es reflexiva $\Leftrightarrow (x,x) \in R \Leftrightarrow \forall x \in U, xRx$

\subsection{Simétrica}

$R$ es simétrica $\Leftrightarrow R^{-1} \subset R \Leftrightarrow \forall x,y \in U, xRy \Rightarrow yRx$.

\subsection{Antisimétrica}

$R$ es antisimétrica $\Leftrightarrow R^{-1} \cap R = \{(x,x)|x \in U\} \Leftrightarrow \forall x,y \in U, xRy \wedge yRx \Rightarrow x=y$.

\subsection{Transitiva}

$R$ es transitiva $\Leftrightarrow R \circ R = R \Leftrightarrow \forall x,y,z \in U, xRy \wedge yRz \Rightarrow xRz$.

\section{Relación de equivalencia}

Una relación $\mathcal{E}$ en el conjunto $U$ se denomina \textbf{relación de equivalencia} si posee las siguientes características:
\begin{enumerate}
\item Reflexiva.
\item Simétrica.
\item Transitiva.
\end{enumerate}

\subsection{Clase de equivalencia}

Dada una relación de equivalencia $\mathcal{E}$ en el conjunto $U$, se denomina \textbf{clase de equivalencia} del elemento $x \in U$ al conjunto imagen de $x$:
\[
x\mathcal{E} = [x] = \{y \in U | x \mathcal{E} y\}
\]

Tiene las siguientes propiedades:
\begin{itemize}
\item Es independiente del representate elegido $x\mathcal{E}y \Rightarrow [x]=[y]$.
\item Cualquier $y \in [x]$ es denominado \textbf{representante de la clase} $[x]$.
\item Son disjuntos: $x \not \mathcal{E}y \Rightarrow [x] \cap [y] = \emptyset$
\end{itemize}

\subsection{Partición de un conjunto}

Una \textbf{partición de un conjunto} $U$ es una familia $P$ de subconjuntos no vacíos de $U$ disjuntos dos a dos y cuya unión es el conjunto $U$. Es decir:
\begin{enumerate}
\item $\forall A,B \in P,A=B \vee A \cap B = \emptyset$.
\item $\bigcup_{A \in P} A = U$.
\end{enumerate}

\subsection{Conjunto cociente}

Dada una relación de equivalencia $\mathcal{E}$ en el conjunto $U$, se denomina \textbf{conjunto cociente}, y se denota por $U/\mathcal{E}$, al conjunto de todas las clases que general la relación de equivalencia $\mathcal{E}$.

$U/\mathcal{E}$ es una partición del conjunto $U$.

\section{Relación de orden}

Una relación $\mathcal{R}$ en el conjunto $U$ se denomina \textbf{relación de orden} is posee las propiedades:
\begin{enumerate}
\item Reflexiva.
\item Antisimétrica.
\item Transitiva.
\end{enumerate}

$\mathcal{R}$ es una \textbf{relación de orden total} si
\[
\mathcal{R}^{-1} \cup \mathcal{R} = U \times U \Leftrightarrow \forall x,y \in U, x\mathcal{R}y \vee y\mathcal{R}x
\]
En cualquier otro caso es una \textbf{relación de orden parcial}.

\subsection{Intervalos en un conjunto ordenado}

Dados un conjunto ordenado $(U,\preceq)$, y $a,b \in U$ tales que $a \preccurlyeq b$, se denominan intervalos a cada uno de los siguientes conjuntos:
\begin{description}
\item[Intervalo abierto] $(a,b) = \{x \in U| a \prec x \prec y\}$
\item[Intervalo cerrado] $[a,b] = \{x \in U| a \preceq x \preceq y\}$
\item[Intervalo semiabiertos] Pueden ser:
\begin{itemize}
\item $(a,b] = \{x \in U| a \prec x \preceq y\}$
\item $[a,b) = \{x \in U| a \preceq x \prec y\}$
\end{itemize}
\end{description}

\subsection{Intervalos iniciales y finales}

Dado un conjunto ordenado $(U,\preceq)$, los siguientes conjuntos también se denominan intervalos:
\begin{description}
\item[Intervalo inicial abierto] $(\leftarrow,a) = \{x \in U| x \prec a\}$.
\item[Intervalo final abierto] $(a,\leftarrow) = \{x \in U| a \prec x\}$.
\item[Intervalo inicial cerrado] $(\leftarrow,a] = \{x \in U| x \preceq a\}$.
\item[Intervalo final cerrado] $[a,\leftarrow) = \{x \in U| a \preceq x\}$.
\end{description}

\subsection{Orden lexicográfico en $\mathbb{R}^2$}

\[
(a,b) \leq_L (c,d) \Leftrightarrow (a < c) \vee (a = c \wedge (b \leq d))
\]

\subsection{Orden producto en $\mathbb{R}^2$}

\[
(a,b) \leq_P (c,d) \Leftrightarrow a \leq c \wedge b \leq d
\]

\subsection{Conjunto acotado}

Dados un conjunto ordenado $(U,\preceq)$ y un subconjunto $A \subset U$, se denomina:
\begin{description}
\item[Cota superior] Cualquier elemento $u \in U$ que cumple que $\forall  x \in A, x \preceq u$.
\item[Cota inferior] Cualquier elemento $u \in U$ que cumple que $\forall  x \in A, u \preceq x$.
\item[Conjunto $A$ acotado superiormente] Si existe una cota superior de $A$.
\item[Conjunto $A$ acotado inferiormente] Si existe una cota inferior de $A$.
\item[Conjunto $A$ acotado] Si existe tanto cota superior como inferior de $A$.
\end{description}

\subsection{Máximo, mínimo, supremo e ínfimo}
\begin{description}
\item[Máximo del conjunto $A$] $\max(A)= m \in A$ tal que $\forall x \in A, x \preceq m$.
\item[Mínimo del conjunto $A$] $\min(A)= m \in A$ tal que $\forall x \in A, m \preceq x$.
\item[Supremo del conjunto $A$] Cota superior $\sup(A)=s \in U$ tal que $s \preceq u$ para toda cota superior $u$ de $A$.
\item[Ínfimo del conjunto $A$] Cota inferior $\inf(A)=i \in U$ tal que $u \preceq i$ para toda cota inferior $u$ de $A$.
\end{description}

El ínfimo es el máximo de las cotas inferiores. El supremo es el mínimo de las cotas superiores.

Si existe máximo entonces hay supremo y son iguales. Si existe mínimo entonces hay ínfimo y son iguales.

Dados un conjunto ordenado $(U,\preceq)$ y un subconjunto $A \subset U$, se tiene:
\begin{itemize}
\item Si existe máximo, mínimo, del conjunto $A$, entonces éste es único.
\item Si existe supremo, ínfimo, del conjunto $A$, entonces éste es único.
\item Si existe supremo $s$ del conjunto $A$ y $s \in A$, entonces $s$ es el máximo de $A$.
\item Si existe ínfimo $i$ del conjunto $A$ y $i \in A$, entonces $i$ es el mínimo de $A$.
\end{itemize}

\subsection{Propiedad del buen orden}

Se dice que un conjunto $(U,\preceq)$ es un conjunto \textbf{bien ordenado}, o que la relación $\preceq$ es una buena ordenación, si cualquier subconjunto no vacío posee mínimo. El elemento mínimo de cada subconjunto $A$ también se denomina primer elemento.

\subsection{Propiedad del supremo}

Se dice que un conjunto $(U,\preceq)$ cumple la propiedad del supremo si y sólo si cualquier subconjunto no vacío $A$ acotado superiormente posee supremo.

\subsection{Maximal y minimal}

Dados un conjunto ordenado $(U,\preceq)$ y un subconjunto $A \subset U$ se define:
\begin{description}
\item[Maximal del conjunto $A$] Es un elemento $m \in A$ tal que
\[
\nexists x \in A, x \neq m, m \preceq x
\] 
\item[Minimal del conjunto $A$] Es un elemento $m \in A$ tal que
\[
\nexists x \in A, x \neq m, x \preceq m
\] 
\end{description}

Si un conjunto tiene máximo (mínimo) entonces solo hay un maximal (minimal) que coincide con él.

\section{Aplicaciones entre conjuntos}

Una relación entre los conjuntos $A$ y $B$ se denomina \textbf{aplicación}, o \textbf{función}, entre $A$ y $B$ si y sólo si cualquier elemento del conjunto inicial $A$ esta relacionado con un único elemento del conjunto final $B$.

\[
\forall a \in A, \exists b \in B; f(a)=b; b=f(a) \wedge b'=f(a) \Rightarrow b=b'
\]

\begin{itemize}
\item $A$ es el conjunto \textbf{inicial, original o dominio de definición} de $f$, y se denota como $Orig(f), Dom(f)$.
\item $B$ es el conjunto \textbf{final} de $f$.
\item $f(A)=Im(f)=\{y \in B|\exists x \in A, f(x)=y\}=\{f(x)| x \in A\}$ es el \textbf{conjunto imagen}.
\item $f(x)$ es la imagen de $x$.
\item $f^{-1}(y)=\{x \in A | f(x)=y\}$ se denomina \textbf{imagen inversa de $y$ por $f$}.
\item El conjunto de aplicaciones de $A$ a $B$ se denota por $\mathcal{F}(A,B),B^A,\mathcal{F}(A)$.
\end{itemize}

Una aplicación $f$ es \textbf{constante}
\[
\Leftrightarrow \forall x,x' \in A, f(x)=f(x')
\] 

Una relación de equivalencia $\mathcal{E}$ sobre un conjunto $A$ define una aplicación $f$ que asigna a cada elemento su clase de equivalencia. Se denomina \textbf{proyección canónica}.

La \textbf{aplicación de identidad} asigna a cada $x \in A$ el mismo valor. Se denota como $I_A,1_A,Id_A$.

\subsection{Igualdad entre aplicaciones}

Dos aplicaciones $f: A \longrightarrow B$ y $g: A' \longrightarrow B'$ son iguales:
\[
f=g \Leftrightarrow \begin{cases}
A=A'\\
B=B'\\
f(x)=g(x)\ \forall x \in A
\end{cases}
\]

\subsection{Composición de aplicaciones}

Dadas las aplicaciones $f: A \longrightarrow B$ y $g: B \longrightarrow C$, se define la \textbf{composición de} f y g, o aplicación composición, a la aplicación de $A$ a $C$, que denotamos como $g \circ f$, tal que:
\[
(g \circ f)(x) = g(f(x)), \forall x \in A
\]

\subsection{Función característica de un conjunto}

Dado un subconjunto $A \subset U$, se llama función característica de $A$, y se denota $\mathcal{X}_A$, a la función $\mathcal{X}_A: U \longrightarrow \mathbb{R}$ definida de la forma:
\[
\mathcal{X}_A = \begin{cases}
1 \; x \in A \\
0 \; x \not \in A
\end{cases}
\]

\subsection{Aplicación sobreyectiva, sobreyección o epiyectiva}

Es una aplicación tal que cualquier elemento del conjunto final está relacionado con alguno del conjunto inicial. Es decir, $f \in \mathcal{F}(A,B)$ tal que $Im(f)=B$, o lo que es lo mismo:
\[
\forall y \in B, \exists x \in A; f(x) = y
\]

\subsection{Aplicación inyectiva o inyección}

Es una aplicación tal que no hay dos elementos del conjunto inicial que tengan la misma imagen. Es decir, $f \in \mathcal{F}(A,B)$ tal que:
\[
\forall x, x' \in A; f(x)=f(x') \Rightarrow x=x'
\]

O lo que es lo mismo:
\[
\forall x, x' \in A; x \neq x' \Rightarrow f(x) \neq f(x')
\]

\subsection{Composición de aplicaciones sobreyectivas e inyectivas}

Dadas las aplicaciones $f \in \mathcal{F}(A,B)$, $g \in \mathcal{F}(B,C)$ y  $g \circ f \in \mathcal{F}(A,B)$, se tiene:
\begin{enumerate}
\item Si $f$ y $g$ son sobreyectivas, entonces $g \circ f$ es sobreyectiva.
\item Si $f$ y $g$ son inyectivas, entonces $g \circ f$ es inyectiva.
\end{enumerate}

\subsection{Aplicación biyectiva o biyección}

Es una aplicación que es sobreyectvia o inyectiva al mismo tiempo, es decir, tal que cada elemento del conjunto final está relacionado cono un único elemento del conjunto inicial. Es decir,  una aplicación $f \in \mathcal{F}(A,B)$ tal que:
\[
para\ cada\ y \in B\ existe\ un\ unico\ elemento\ x \in A\ tal\ que\ f(x)=y
\]

Una aplicación $f \in \mathcal{F}(A,B)$ es biyectiva si y sólo si existe una aplicación $g \in \mathcal{F}(B,A)$ tal que $f \circ g = I_B$ y $g \circ f = I_A$.

Sean $f \in \mathcal{F}(A,B)$ y $g \in \mathcal{F}(B,C)$ aplicaciones biyectivas, entonces $g \circ f \in \mathcal{F}(A,C)$ es biyectiva, y su inversa es:
\[
(g \circ f)^{-1}(x)=f^{-1}(x) \circ g^{-1}(x)
\]

Sea una aplicación $f \in \mathcal{F}(A,B)$:
\begin{enumerate}
\item $f$ es sobreyectiva si y sólo si existe una aplicación $h \in \mathcal{F}(B,A)$ tal que $f \circ h = I_B$.
\item $f$ es inyectiva si y sólo si existe una aplicación $g \in \mathcal{F}(A,B)$ tal que $g \circ f = I_A$.
\end{enumerate}

\section{Equipotencia de conjuntos}

Dos conjuntos $A$ y $B$ se dicen equipotentes si y sólo si existe una biyección entre ellos, y se denota $A \equiv B$.

\subsection{Cardinal}

Se denomina:
\begin{description}
\item[Cardinal 0] Es la colección de todos los conjuntos equipotentes con $\emptyset$, y se representa con el símbolo del número 0.
\item[Cardinal n] Es la colección de todos los conjuntos equipotentes con ${1,\ldots,n} \subset \mathbb{N}^*$, y se representa con el símbolo del número $n$.
\item[Cardinal de $\mathbb{N}$ o $\mathfrak{N}_0$] Es la colección de todos los conjuntos equipotentes con $\mathbb{N}$, y se representa con el símbolo del número $\mathfrak{N}_0$.
\item[Cardinal $\mathbb{R}$ o $\mathfrak{c}$] Es la colección de todos los conjuntos equipotentes con $\mathbb{R}$, y se representa con el símbolo del número $\mathfrak{c}$.
\end{description}

Decimos que el conjunto $A$ tiene $n$ elementos siendo $n \in \mathbb{N}^*$ si y sólo si:
\[
card(A)=n
\]

Se dice que un conjunto $A$ es:
\begin{description}
\item[Finito] Si existe $n \in \mathbb{N}$ tal que $card(A)=n$.
\item[Infinito] Si no es un conjunto finito.
\item[Numerable] Si existe una biyección de los números naturales al conjunto, y se indica escribiendo $card(A)=\mathfrak{N}_0$.
\end{description}