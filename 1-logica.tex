\chapter{Nociones de lógica}

\section{Conectores lógicos básicos}

\subsection{Negación}

\begin{table}[htbp]
  \centering
  \begin{tabular}{|l|l|}
    \hline
    $p$ & $\neg p$ \\
    \hline
    0 & 1 \\
    1 & 0 \\
    \hline
  \end{tabular}
  \caption{Negación}
\end{table}

\subsection{Disyunción}

\begin{table}[htbp]
  \centering
  \begin{tabular}{|l|l|l|}
    \hline
    $p$ & $q$ & $p \vee q$ \\
    \hline
    0 & 0 & 0\\
    0 & 1 & 1\\
    1 & 0 & 1\\
    1 & 1 & 1\\
    \hline
  \end{tabular}
  \caption{Disyunción}
\end{table}

\subsection{Conjunción}

\begin{table}[htbp]
  \centering
  \begin{tabular}{|l|l|l|}
    \hline
    $p$ & $q$ & $p \wedge q$ \\
    \hline
    0 & 0 & 0\\
    0 & 1 & 0\\
    1 & 0 & 0\\
    1 & 1 & 1\\
    \hline
  \end{tabular}
  \caption{Conjunción}
\end{table}

\subsection{Condicional}

\begin{table}[htbp]
  \centering
  \begin{tabular}{|l|l|l|}
    \hline
    $p$ & $q$ & $p \rightarrow q$ \\
    \hline
    0 & 0 & 1\\
    0 & 1 & 1\\
    1 & 0 & 0\\
    1 & 1 & 1\\
    \hline
  \end{tabular}
  \caption{Disyunción}
\end{table}

A la proposicional condicional se le asocian 3 nuevas proposiciones:
\begin{description}
 \item[Condiconal recíproco] $q \rightarrow p$.
 \item[Condiconal contrario] $\neg p \rightarrow \neg q$.
 \item[Condiconal contrarrecíproco] $\neg q \rightarrow \neg p$.
\end{description}

\subsection{Bicondicional}

\begin{table}[htbp]
  \centering
  \begin{tabular}{|l|l|l|}
    \hline
    $p$ & $q$ & $p \leftrightarrow q$ \\
    \hline
    0 & 0 & 1\\
    0 & 1 & 0\\
    1 & 0 & 0\\
    1 & 1 & 1\\
    \hline
  \end{tabular}
  \caption{Disyunción}
\end{table}

\section{Construcción de nuevas propociones}

\begin{description}
 \item[Contradicción] Proposición que solo toma el valor 0, se denota como 
\textbf{0}.
 \item[Tautología] Proposición que solo toma el valor 1, se denota como 
\textbf{1}.
\end{description}

\section{Leyes lógicas}

\subsection{Leyes lógicas con una proposición}

\subsection{Doble negación}

\[
 \neg \neg p \Leftrightarrow p
\]


\subsection{Simplificación}

\begin{enumerate}
 \item $p \vee p \Leftrightarrow p$
 \item $p \wedge p \Leftrightarrow p$
 \item $p \rightarrow p \Leftrightarrow p$
 \item $p \leftrightarrow p \Leftrightarrow p$
\end{enumerate}

\subsection{Tercio exclusivo}

\[
 p \vee \neg p \Leftrightarrow \textbf{1}
\]

\subsection{Contradicción}

\[
 p \wedge \neg p \Leftrightarrow \textbf{0}
\]

\subsection{Leyes lógicas equivalentes con dos proposiciones}

\subsection{Identidad}

\begin{enumerate}
 \item $p \vee 0 \Leftrightarrow p$ y $p \vee 1 \Leftrightarrow 1$.
 \item $p \wedge 1 \Leftrightarrow p$ y $p \wedge 0 \Leftrightarrow 0$.
 \item $1 \rightarrow p \Leftrightarrow p$.
\end{enumerate}

\subsection{De Morgan}

\begin{enumerate}
 \item $\neg (p \vee q) \Leftrightarrow \neg p \wedge \neg q$.
 \item $\neg (p \wedge q) \Leftrightarrow \neg p \vee \neg q$.
\end{enumerate}

\subsection{Del condicional}

\begin{enumerate}
 \item $p \rightarrow q \Leftrightarrow \neg p \vee q$.
 \item $p \rightarrow q \Leftrightarrow \neg (p \wedge \neg q)$.
 \item $p \rightarrow q \Leftrightarrow p \rightarrow (p \wedge q)$.
 \item $p \rightarrow q \Leftrightarrow q \rightarrow (p \vee q)$.
\end{enumerate}

\subsection{Del bicondicional}

\[
 p \leftrightarrow q \Leftrightarrow (p \rightarrow q) \wedge (q \rightarrow p)
\]

\subsection{Reducción al absurdo}

\[
 \neg p \rightarrow (q \wedge \neg q) \Leftrightarrow p
\]

\subsection{Transposición}

\begin{enumerate}
 \item $p \rightarrow q \Leftrightarrow \neg q \rightarrow \neg p$.
 \item $p \leftrightarrow q \Leftrightarrow \neg p \leftrightarrow \neg q$.
\end{enumerate}

\subsection{Leyes lógicas equivalentes con tres proposiciones}

\subsection{Asociativas}

\begin{enumerate}
 \item $p \vee (q \vee r) \Leftrightarrow (p \vee q) \vee r$.
 \item $p \wedge (q \wedge r) \Leftrightarrow (p \wedge q) \wedge r$.
 \item $p \leftrightarrow (q \leftrightarrow r) \Leftrightarrow (p 
\leftrightarrow q) \leftrightarrow r$.
 \item $p \rightarrow (q \vee r) \Leftrightarrow (p \rightarrow q) \vee (p 
\rightarrow q)$.
 \item $p \rightarrow (q \wedge r) \Leftrightarrow (p \rightarrow q) \wedge (p 
\rightarrow q)$.
\end{enumerate}

\subsection{Distributivas}

\begin{enumerate}
\item $p \wedge (q \vee r) \Leftrightarrow (p \wedge q) \vee (p 
\wedge q)$.
\item $p \vee (q \wedge r) \Leftrightarrow (p \vee q) \wedge (p 
\vee q)$.
 \item $p \rightarrow (q \vee r) \Leftrightarrow (p \rightarrow q) \vee (p 
\rightarrow q)$.
 \item $p \rightarrow (q \wedge r) \Leftrightarrow (p \rightarrow q) \wedge (p 
\rightarrow q)$.
\end{enumerate}

\subsection{Leyes lógicas condicionales}

\subsection{Simplificación condicional}

\begin{enumerate}
 \item $p \wedge q \Rightarrow p$.
 \item $p \Rightarrow p \vee q$
\end{enumerate}

\subsection{Inferencia}

\begin{enumerate}
 \item $\neg p \wedge(p \vee q) \Rightarrow q$.
 \item $p \wedge (\neg p \vee \neg q) \Rightarrow \neg q$.
\end{enumerate}

\subsection{Ponendo ponens}

\[
 (p \rightarrow q) \wedge p \Rightarrow q
\]

\subsection{Tollendo tollens}

\[
 (p \rightarrow q) \wedge \neg q \Rightarrow \neg p
\]


\subsection{Validación de proposiciones}

La validación de las proposiciones se puede realizar mediante:
\begin{itemize}
 \item Construcción de la tabla de verdad.
 \item Refutación: Se aplica reducción al absurdo.
\end{itemize}

\section{Forma clausulada}

Proposición formada por la conjunción ($\wedge$) de disyunciones ($\vee$).